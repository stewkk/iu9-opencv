% Options for packages loaded elsewhere
\PassOptionsToPackage{unicode}{hyperref}
\PassOptionsToPackage{hyphens}{url}
\PassOptionsToPackage{dvipsnames,svgnames,x11names}{xcolor}
%
\documentclass[
  12pt,
  a4paper,
]{article}
\usepackage{amsmath,amssymb}
\usepackage{lmodern}
\usepackage{iftex}
\ifPDFTeX
  \usepackage[T1]{fontenc}
  \usepackage[utf8]{inputenc}
  \usepackage{textcomp} % provide euro and other symbols
\else % if luatex or xetex
  \usepackage{unicode-math}
  \defaultfontfeatures{Scale=MatchLowercase}
  \defaultfontfeatures[\rmfamily]{Ligatures=TeX,Scale=1}
\fi
% Use upquote if available, for straight quotes in verbatim environments
\IfFileExists{upquote.sty}{\usepackage{upquote}}{}
\IfFileExists{microtype.sty}{% use microtype if available
  \usepackage[]{microtype}
  \UseMicrotypeSet[protrusion]{basicmath} % disable protrusion for tt fonts
}{}
\makeatletter
\@ifundefined{KOMAClassName}{% if non-KOMA class
  \IfFileExists{parskip.sty}{%
    \usepackage{parskip}
  }{% else
    \setlength{\parindent}{0pt}
    \setlength{\parskip}{6pt plus 2pt minus 1pt}}
}{% if KOMA class
  \KOMAoptions{parskip=half}}
\makeatother
\usepackage{xcolor}
\IfFileExists{xurl.sty}{\usepackage{xurl}}{} % add URL line breaks if available
\IfFileExists{bookmark.sty}{\usepackage{bookmark}}{\usepackage{hyperref}}
\hypersetup{
  colorlinks=true,
  linkcolor={Maroon},
  filecolor={Maroon},
  citecolor={Blue},
  urlcolor={Blue},
  pdfcreator={LaTeX via pandoc}}
\urlstyle{same} % disable monospaced font for URLs
\usepackage{color}
\usepackage{fancyvrb}
\newcommand{\VerbBar}{|}
\newcommand{\VERB}{\Verb[commandchars=\\\{\}]}
\DefineVerbatimEnvironment{Highlighting}{Verbatim}{commandchars=\\\{\}}
% Add ',fontsize=\small' for more characters per line
\newenvironment{Shaded}{}{}
\newcommand{\AlertTok}[1]{\textcolor[rgb]{1.00,0.00,0.00}{\textbf{#1}}}
\newcommand{\AnnotationTok}[1]{\textcolor[rgb]{0.38,0.63,0.69}{\textbf{\textit{#1}}}}
\newcommand{\AttributeTok}[1]{\textcolor[rgb]{0.49,0.56,0.16}{#1}}
\newcommand{\BaseNTok}[1]{\textcolor[rgb]{0.25,0.63,0.44}{#1}}
\newcommand{\BuiltInTok}[1]{#1}
\newcommand{\CharTok}[1]{\textcolor[rgb]{0.25,0.44,0.63}{#1}}
\newcommand{\CommentTok}[1]{\textcolor[rgb]{0.38,0.63,0.69}{\textit{#1}}}
\newcommand{\CommentVarTok}[1]{\textcolor[rgb]{0.38,0.63,0.69}{\textbf{\textit{#1}}}}
\newcommand{\ConstantTok}[1]{\textcolor[rgb]{0.53,0.00,0.00}{#1}}
\newcommand{\ControlFlowTok}[1]{\textcolor[rgb]{0.00,0.44,0.13}{\textbf{#1}}}
\newcommand{\DataTypeTok}[1]{\textcolor[rgb]{0.56,0.13,0.00}{#1}}
\newcommand{\DecValTok}[1]{\textcolor[rgb]{0.25,0.63,0.44}{#1}}
\newcommand{\DocumentationTok}[1]{\textcolor[rgb]{0.73,0.13,0.13}{\textit{#1}}}
\newcommand{\ErrorTok}[1]{\textcolor[rgb]{1.00,0.00,0.00}{\textbf{#1}}}
\newcommand{\ExtensionTok}[1]{#1}
\newcommand{\FloatTok}[1]{\textcolor[rgb]{0.25,0.63,0.44}{#1}}
\newcommand{\FunctionTok}[1]{\textcolor[rgb]{0.02,0.16,0.49}{#1}}
\newcommand{\ImportTok}[1]{#1}
\newcommand{\InformationTok}[1]{\textcolor[rgb]{0.38,0.63,0.69}{\textbf{\textit{#1}}}}
\newcommand{\KeywordTok}[1]{\textcolor[rgb]{0.00,0.44,0.13}{\textbf{#1}}}
\newcommand{\NormalTok}[1]{#1}
\newcommand{\OperatorTok}[1]{\textcolor[rgb]{0.40,0.40,0.40}{#1}}
\newcommand{\OtherTok}[1]{\textcolor[rgb]{0.00,0.44,0.13}{#1}}
\newcommand{\PreprocessorTok}[1]{\textcolor[rgb]{0.74,0.48,0.00}{#1}}
\newcommand{\RegionMarkerTok}[1]{#1}
\newcommand{\SpecialCharTok}[1]{\textcolor[rgb]{0.25,0.44,0.63}{#1}}
\newcommand{\SpecialStringTok}[1]{\textcolor[rgb]{0.73,0.40,0.53}{#1}}
\newcommand{\StringTok}[1]{\textcolor[rgb]{0.25,0.44,0.63}{#1}}
\newcommand{\VariableTok}[1]{\textcolor[rgb]{0.10,0.09,0.49}{#1}}
\newcommand{\VerbatimStringTok}[1]{\textcolor[rgb]{0.25,0.44,0.63}{#1}}
\newcommand{\WarningTok}[1]{\textcolor[rgb]{0.38,0.63,0.69}{\textbf{\textit{#1}}}}
\setlength{\emergencystretch}{3em} % prevent overfull lines
\providecommand{\tightlist}{%
  \setlength{\itemsep}{0pt}\setlength{\parskip}{0pt}}
\setcounter{secnumdepth}{-\maxdimen} % remove section numbering
\ifLuaTeX
  \usepackage{selnolig}  % disable illegal ligatures
\fi

\newcommand{\thetitle}{Введение в CV на примере реализации задачи Key
point detection на C++ и Python}
\newcommand{\theauthor}{Старовойтов А. И.}
\newcommand{\theteacher}{Посевин Д. П.}
\newcommand{\thegroup}{ИУ9-21Б}
\newcommand{\thecourse}{Языки и методы программирования}
\newcommand{\thenumber}{2}

\usepackage{geometry}

\usepackage{fontspec}
\setmainfont{DejaVu Serif}
\setsansfont{DejaVu Sans}
\setmonofont{DejaVu Sans Mono}
\defaultfontfeatures{Ligatures=TeX}
\usepackage{polyglossia}
\usepackage[autostyle=true]{csquotes}
\setdefaultlanguage{russian}
\setotherlanguage{english}

\usepackage{fvextra}

\renewcommand{\maketitle}
{
\newgeometry{
  left=0.7in,
  right=0.7in,
}
\begin{titlepage}
    \centering
    Федеральное государственное бюджетное образовательное учреждение\\
    высшего профессионального образования\\
    <<Московский государственный технический университет\\
    имени Н.Э. Баумана>>\\
    (МГТУ им. Н.Э. Баумана)
    \vspace{1cm}

    \flushleft

    Факультет: \underline{Информатика и системы управления}\\
    Кафедра: \underline{Теоретическая информатика и компьютерные технологии}

    \centering
    \topskip0pt
    \vspace*{\fill}
    ДЗ №\thenumber{}\\
    <<\thetitle{}>>\\
    по курсу: <<\thecourse{}>>
    \vspace*{\fill}
    \centering

    %\vspace{2cm}

    \hfill\begin{minipage}{0.4\linewidth}
        Выполнил:\\
        Студент группы \thegroup{}\\
        \theauthor\\
        \\
        Проверил:\\
        \theteacher
    \end{minipage}

    \vfill

    Москва, \the\year{}

\end{titlepage}
\restoregeometry{}
}

\usepackage{fvextra}
\DefineVerbatimEnvironment{Highlighting}{Verbatim}{breaklines,commandchars=\\\{\}}

\begin{document}
\maketitle

\hypertarget{ux446ux435ux43bux438}{%
\section{Цели}\label{ux446ux435ux43bux438}}

Знакомство с возможностями языка С++ и Python для реализации задач
машинного зрения.

\hypertarget{ux437ux430ux434ux430ux447ux438}{%
\section{Задачи}\label{ux437ux430ux434ux430ux447ux438}}

Реализовать на C++ (см. п. 2.1.) и Python (см. п. 2.2.) под любую ОС по
желанию студента следующие задачи:

\begin{enumerate}
\def\labelenumi{\arabic{enumi}.}
\tightlist
\item
  Распознавание координат точек кисти со снимков получаемых с камеры,
  координаты точек выводятся списком в консоль в формате JSON.
\item
  Распознавание координат точек тела со снимков получаемых с камеры,
  координаты точек выводятся списком в консоль в формате JSON.
\item
  Сравнить скорость работы алгоритма распознавания кисти руки
  выполненного на C++ со скоростью распознавания выполненного на Python.
  В отчете привести сравнение скоростей.
\item
  Сравнить скорость распознавания кисти руки алгоритмом выполненным на
  языке Python в этом Модуле со скоростью алгоритма распознавания кисти
  руки на базе Mediapipe выполненным на языке Python в предыдущем Модуле
  №1. В отчете привести сравнение скоростей.
\item
  Сделать выводы.
\end{enumerate}

\hypertarget{ux440ux435ux448ux435ux43dux438ux435}{%
\section{Решение}\label{ux440ux435ux448ux435ux43dux438ux435}}

\hypertarget{ux440ux430ux441ux43fux43eux437ux43dux43eux432ux430ux43dux438ux435-ux43aux438ux441ux442ux438}{%
\subsection{Распознование
кисти}\label{ux440ux430ux441ux43fux43eux437ux43dux43eux432ux430ux43dux438ux435-ux43aux438ux441ux442ux438}}

\begin{verbatim}
python handPoseImage.py
time taken by network : 0.510
Total time taken : 0.620
\end{verbatim}

\begin{verbatim}
./handPoseImage
Time Taken = 0.7435
\end{verbatim}

Код для вывода в \texttt{JSON} на \texttt{Python}:

\begin{Shaded}
\begin{Highlighting}[]
\ControlFlowTok{with} \BuiltInTok{open}\NormalTok{(}\StringTok{\textquotesingle{}output.json\textquotesingle{}}\NormalTok{, }\StringTok{\textquotesingle{}w\textquotesingle{}}\NormalTok{) }\ImportTok{as}\NormalTok{ outfile:}
\NormalTok{    json.dump(\{}\StringTok{"points"}\NormalTok{: points\}, outfile, ensure\_ascii}\OperatorTok{=}\VariableTok{False}\NormalTok{, indent}\OperatorTok{=}\DecValTok{4}\NormalTok{)}
\end{Highlighting}
\end{Shaded}

Пример \texttt{JSON} вывода для \texttt{Python}:

\begin{Shaded}
\begin{Highlighting}[]
\FunctionTok{\{}
    \DataTypeTok{"points"}\FunctionTok{:} \OtherTok{[}
        \OtherTok{[}
            \DecValTok{263}\OtherTok{,}
            \DecValTok{638}
        \OtherTok{],}
        \OtherTok{[}
            \DecValTok{365}\OtherTok{,}
            \DecValTok{570}
        \OtherTok{],}
        \OtherTok{[}
            \DecValTok{450}\OtherTok{,}
            \DecValTok{519}
        \OtherTok{],}
        \OtherTok{[}
            \DecValTok{518}\OtherTok{,}
            \DecValTok{434}
        \OtherTok{],}
        \OtherTok{[}
            \DecValTok{518}\OtherTok{,}
            \DecValTok{349}
        \OtherTok{],}
        \OtherTok{[}
            \DecValTok{348}\OtherTok{,}
            \DecValTok{366}
        \OtherTok{],}
        \OtherTok{[}
            \DecValTok{365}\OtherTok{,}
            \DecValTok{230}
        \OtherTok{],}
        \OtherTok{[}
            \DecValTok{382}\OtherTok{,}
            \DecValTok{161}
        \OtherTok{],}
        \OtherTok{[}
            \DecValTok{382}\OtherTok{,}
            \DecValTok{76}
        \OtherTok{],}
        \OtherTok{[}
            \DecValTok{280}\OtherTok{,}
            \DecValTok{366}
        \OtherTok{],}
        \OtherTok{[}
            \DecValTok{297}\OtherTok{,}
            \DecValTok{212}
        \OtherTok{],}
        \OtherTok{[}
            \DecValTok{280}\OtherTok{,}
            \DecValTok{111}
        \OtherTok{],}
        \OtherTok{[}
            \DecValTok{280}\OtherTok{,}
            \DecValTok{25}
        \OtherTok{],}
        \OtherTok{[}
            \DecValTok{229}\OtherTok{,}
            \DecValTok{383}
        \OtherTok{],}
        \OtherTok{[}
            \DecValTok{212}\OtherTok{,}
            \DecValTok{230}
        \OtherTok{],}
        \OtherTok{[}
            \DecValTok{195}\OtherTok{,}
            \DecValTok{161}
        \OtherTok{],}
        \OtherTok{[}
            \DecValTok{195}\OtherTok{,}
            \DecValTok{76}
        \OtherTok{],}
        \OtherTok{[}
            \DecValTok{161}\OtherTok{,}
            \DecValTok{400}
        \OtherTok{],}
        \OtherTok{[}
            \DecValTok{127}\OtherTok{,}
            \DecValTok{315}
        \OtherTok{],}
        \OtherTok{[}
            \DecValTok{127}\OtherTok{,}
            \DecValTok{247}
        \OtherTok{],}
        \OtherTok{[}
            \DecValTok{110}\OtherTok{,}
            \DecValTok{161}
        \OtherTok{],}
        \KeywordTok{null}
    \OtherTok{]}
\FunctionTok{\}}
\end{Highlighting}
\end{Shaded}

Код для вывода в \texttt{JSON} на \texttt{C++}:

\begin{Shaded}
\begin{Highlighting}[]
\NormalTok{    FileStorage fs}\OperatorTok{(}\StringTok{"output.json"}\OperatorTok{,}\NormalTok{ FileStorage}\OperatorTok{::}\NormalTok{WRITE}\OperatorTok{);}

\NormalTok{    fs }\OperatorTok{\textless{}\textless{}} \StringTok{"points"} \OperatorTok{\textless{}\textless{}} \StringTok{"["}\OperatorTok{;}
    \ControlFlowTok{for} \OperatorTok{(}\KeywordTok{auto}\OperatorTok{\&}\NormalTok{ el }\OperatorTok{:}\NormalTok{ points}\OperatorTok{)} \OperatorTok{\{}
\NormalTok{        fs }\OperatorTok{\textless{}\textless{}}\NormalTok{ el}\OperatorTok{;}
    \OperatorTok{\}}
\NormalTok{    fs }\OperatorTok{\textless{}\textless{}} \StringTok{"]"}\OperatorTok{;}
\end{Highlighting}
\end{Shaded}

Пример \texttt{JSON} вывода для \texttt{C++}:

\begin{Shaded}
\begin{Highlighting}[]
\FunctionTok{\{}
    \DataTypeTok{"points"}\FunctionTok{:} \OtherTok{[}
        \OtherTok{[} \DecValTok{263}\OtherTok{,} \DecValTok{638} \OtherTok{],}
        \OtherTok{[} \DecValTok{365}\OtherTok{,} \DecValTok{570} \OtherTok{],}
        \OtherTok{[} \DecValTok{450}\OtherTok{,} \DecValTok{519} \OtherTok{],}
        \OtherTok{[} \DecValTok{518}\OtherTok{,} \DecValTok{434} \OtherTok{],}
        \OtherTok{[} \DecValTok{518}\OtherTok{,} \DecValTok{349} \OtherTok{],}
        \OtherTok{[} \DecValTok{348}\OtherTok{,} \DecValTok{366} \OtherTok{],}
        \OtherTok{[} \DecValTok{365}\OtherTok{,} \DecValTok{230} \OtherTok{],}
        \OtherTok{[} \DecValTok{382}\OtherTok{,} \DecValTok{161} \OtherTok{],}
        \OtherTok{[} \DecValTok{382}\OtherTok{,} \DecValTok{76} \OtherTok{],}
        \OtherTok{[} \DecValTok{280}\OtherTok{,} \DecValTok{366} \OtherTok{],}
        \OtherTok{[} \DecValTok{297}\OtherTok{,} \DecValTok{212} \OtherTok{],}
        \OtherTok{[} \DecValTok{280}\OtherTok{,} \DecValTok{111} \OtherTok{],}
        \OtherTok{[} \DecValTok{280}\OtherTok{,} \DecValTok{25} \OtherTok{],}
        \OtherTok{[} \DecValTok{229}\OtherTok{,} \DecValTok{383} \OtherTok{],}
        \OtherTok{[} \DecValTok{212}\OtherTok{,} \DecValTok{230} \OtherTok{],}
        \OtherTok{[} \DecValTok{195}\OtherTok{,} \DecValTok{161} \OtherTok{],}
        \OtherTok{[} \DecValTok{195}\OtherTok{,} \DecValTok{76} \OtherTok{],}
        \OtherTok{[} \DecValTok{161}\OtherTok{,} \DecValTok{400} \OtherTok{],}
        \OtherTok{[} \DecValTok{127}\OtherTok{,} \DecValTok{315} \OtherTok{],}
        \OtherTok{[} \DecValTok{127}\OtherTok{,} \DecValTok{247} \OtherTok{],}
        \OtherTok{[} \DecValTok{110}\OtherTok{,} \DecValTok{161} \OtherTok{],}
        \OtherTok{[} \DecValTok{127}\OtherTok{,} \DecValTok{247} \OtherTok{]}
    \OtherTok{]}
\FunctionTok{\}}
\end{Highlighting}
\end{Shaded}

\hypertarget{ux440ux430ux441ux43fux43eux437ux43dux43eux432ux430ux43dux438ux435-ux442ux435ux43bux430}{%
\subsection{Распознование
тела}\label{ux440ux430ux441ux43fux43eux437ux43dux43eux432ux430ux43dux438ux435-ux442ux435ux43bux430}}

\begin{verbatim}
python3 OpenPoseImage.py 
Using CPU device
time taken by network : 0.979
Total time taken : 1.352
\end{verbatim}

\begin{verbatim}
./OpenPoseImage 
USAGE : ./OpenPose <imageFile> 
USAGE : ./OpenPose <imageFile> <device>
Using CPU device
Time Taken = 1.0332
\end{verbatim}

Код для вывода в \texttt{JSON} одинаков.

Пример вывода в \texttt{JSON} из \texttt{C++}:

\begin{Shaded}
\begin{Highlighting}[]
\FunctionTok{\{}
    \DataTypeTok{"points"}\FunctionTok{:} \OtherTok{[}
        \OtherTok{[} \DecValTok{348}\OtherTok{,} \DecValTok{125} \OtherTok{],}
        \OtherTok{[} \DecValTok{334}\OtherTok{,} \DecValTok{230} \OtherTok{],}
        \OtherTok{[} \DecValTok{250}\OtherTok{,} \DecValTok{271} \OtherTok{],}
        \OtherTok{[} \DecValTok{237}\OtherTok{,} \DecValTok{376} \OtherTok{],}
        \OtherTok{[} \DecValTok{223}\OtherTok{,} \DecValTok{480} \OtherTok{],}
        \OtherTok{[} \DecValTok{417}\OtherTok{,} \DecValTok{271} \OtherTok{],}
        \OtherTok{[} \DecValTok{417}\OtherTok{,} \DecValTok{397} \OtherTok{],}
        \OtherTok{[} \DecValTok{390}\OtherTok{,} \DecValTok{501} \OtherTok{],}
        \OtherTok{[} \DecValTok{292}\OtherTok{,} \DecValTok{501} \OtherTok{],}
        \OtherTok{[} \DecValTok{306}\OtherTok{,} \DecValTok{668} \OtherTok{],}
        \OtherTok{[} \DecValTok{306}\OtherTok{,} \DecValTok{835} \OtherTok{],}
        \OtherTok{[} \DecValTok{348}\OtherTok{,} \DecValTok{501} \OtherTok{],}
        \OtherTok{[} \DecValTok{348}\OtherTok{,} \DecValTok{668} \OtherTok{],}
        \OtherTok{[} \DecValTok{334}\OtherTok{,} \DecValTok{793} \OtherTok{],}
        \OtherTok{[} \DecValTok{334}\OtherTok{,} \DecValTok{376} \OtherTok{]}
    \OtherTok{]}
\FunctionTok{\}}
\end{Highlighting}
\end{Shaded}

Пример вывода в \texttt{JSON} из \texttt{Python}:

\begin{Shaded}
\begin{Highlighting}[]
\FunctionTok{\{}
    \DataTypeTok{"points"}\FunctionTok{:} \OtherTok{[}
        \OtherTok{[}
            \DecValTok{361}\OtherTok{,}
            \DecValTok{187}
        \OtherTok{],}
        \OtherTok{[}
            \DecValTok{333}\OtherTok{,}
            \DecValTok{271}
        \OtherTok{],}
        \OtherTok{[}
            \DecValTok{250}\OtherTok{,}
            \DecValTok{271}
        \OtherTok{],}
        \OtherTok{[}
            \DecValTok{236}\OtherTok{,}
            \DecValTok{375}
        \OtherTok{],}
        \OtherTok{[}
            \DecValTok{208}\OtherTok{,}
            \DecValTok{480}
        \OtherTok{],}
        \OtherTok{[}
            \DecValTok{417}\OtherTok{,}
            \DecValTok{271}
        \OtherTok{],}
        \OtherTok{[}
            \DecValTok{417}\OtherTok{,}
            \DecValTok{396}
        \OtherTok{],}
        \OtherTok{[}
            \DecValTok{389}\OtherTok{,}
            \DecValTok{500}
        \OtherTok{],}
        \OtherTok{[}
            \DecValTok{292}\OtherTok{,}
            \DecValTok{500}
        \OtherTok{],}
        \OtherTok{[}
            \DecValTok{306}\OtherTok{,}
            \DecValTok{667}
        \OtherTok{],}
        \OtherTok{[}
            \DecValTok{306}\OtherTok{,}
            \DecValTok{855}
        \OtherTok{],}
        \OtherTok{[}
            \DecValTok{361}\OtherTok{,}
            \DecValTok{500}
        \OtherTok{],}
        \OtherTok{[}
            \DecValTok{361}\OtherTok{,}
            \DecValTok{667}
        \OtherTok{],}
        \OtherTok{[}
            \DecValTok{333}\OtherTok{,}
            \DecValTok{813}
        \OtherTok{],}
        \OtherTok{[}
            \DecValTok{347}\OtherTok{,}
            \DecValTok{187}
        \OtherTok{],}
        \OtherTok{[}
            \DecValTok{375}\OtherTok{,}
            \DecValTok{166}
        \OtherTok{],}
        \OtherTok{[}
            \DecValTok{306}\OtherTok{,}
            \DecValTok{166}
        \OtherTok{],}
        \KeywordTok{null}
    \OtherTok{]}
\FunctionTok{\}}
\end{Highlighting}
\end{Shaded}

\hypertarget{ux432ux44bux432ux43eux434}{%
\subsection{Вывод}\label{ux432ux44bux432ux43eux434}}

Т.к. \texttt{python-opencv}
\href{https://docs.opencv.org/4.x/da/d49/tutorial_py_bindings_basics.html}{это
лишь ``обертка'' для обычных \texttt{C++} функций} из \texttt{opencv},
разницы в скорости нет.

\end{document}
